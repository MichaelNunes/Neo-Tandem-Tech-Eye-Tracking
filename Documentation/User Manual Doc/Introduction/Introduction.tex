\subsection{Scope and Purpose}
The purpose of this eye-tracking software is to provide the user with a eye-tracking software which will allow for the tracking of the eye on 2D and 3D models and videos.The system is centred around making eye-tracking possible for all media types.

Our client wanted us to develop this software so that it could help the Geo-Informatics department with research they want to do on how people view geological locations and what stands out to them when they view it.That is why we had decided to give the option of three different types of eye-tracking.This software will thus aid them in their research and help provide the data that they needed.

\subsection{Process Overview }
The Eye-tribe as a system comes with the infrared camera which is used to track the eyes and it also comes with an SDK that will allow you to developed programs to be used with the camera.there is also access to a few sample programs one the Eye-Tribe website.these can be downloaded and then they can be used with the camera to see all the functionality it is capable of.

These items found when purchasing the Eye-tribe camera form only the base of what the camera is really capable of.We have taken these elements and expended them and integrated them with aspects such as 3D modeling and tracking eyes on multiple media formats.

There are a few core process that have been implemented and that will allow for more effective eye-tracking across mediums.The process are as follows:
\begin{itemize}
\item Eye tracking on 2D and 3D models and videos.
\item Creation of heat maps from eye-tracking.
\item Saving heat map on specific media.
\item Creating Statistics on information gathered.
\end{itemize}

The eye tracking on the various models uses the camera to track the eyes of the user and then saves the data in a file so that it can be used later.

The data that is saved is then used to create heatmaps.The heatmaps come in two forms:normal and then overlay.Normal is just the basic points of the heatmap on a blank background.Overlay is a heatmap that has been overlaid onto the media and shows where on the media the user had looked.

The recorded data is also used to create a statistical report of the data.This report includes metadata about the media recorded and then actual data from the stored points.The data on the recorded points includes  data such as time,points gathered and a map that shows which sections where most looked at.the report is in PDF format.

The following process listed are the core of this program as they carry out the basic and most important functionality of this software.
