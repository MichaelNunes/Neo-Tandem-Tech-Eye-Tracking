\subsection{Eye tracking on 3D models and videos}
The eye tracking is done with the eye-tribe camera.The camera tracks gaze and eye movements of the user looking at the medium.The tracking of the eye on 3D model and video is not the same as if it were on a static medium such as a image or a sideshow.The user will select the 3D model or video that they eye tracking must be performed on.The user will then begin the eye-tracking and then they application will record the information.

\subsection{Creation of heat maps from eye-tracking}
The creation of the heat map will use the data collected from the eye tracking and then generate a heat map based on the media type.This can easily be done by clicking the "generate heat map" or it can be set to automatically do so in the settings.
\subsubsection{Converting raw information into heat map compatible information}
The information collected from the eye tracker will need to be converted to the correct format so that the OGAMA module can then use to create the heat map.
\subsubsection{Generation of heat map}
The generation of the heat map is handled by the OGAMA module.The module will take in the converted information and then create a heat map.
\subsection{Saving heat map on specific media}
The heat map that is created will be able to be applied to the media that it was created for.This is done after the heat map is created.This feature will allow the user to save a copy of the heat map over the media of choice.The user needs to select the media object and click on the save with heat map overlay to save the item.
\subsubsection{Duplicating the media}
In order to apply the overlay the media needs to be duplicated as to not overwrite the file used in the eye tracking process.Thus the duplication of the media has to occur.
\subsubsection{Applying heat map to media and Saving}
Once the duplication of the media is complete the over lay needs to be applied to it and needs to be viewable when the media is viewed.The user will be prompted to save the newly created media object into the directory of their choosing.
\subsection{Creating Statistics on information gathered}
The user at any time can generate stats on the  media that was tracked.This action will produce a document that can be saved or printed and will allow the user to view stats suched as most view sections and also average time spent looking at sections of the media.
