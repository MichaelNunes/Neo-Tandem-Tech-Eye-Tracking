Usability testing was seen as an essential exercise to ensure that all features for our application were easy to use. 

\subsection{Date of testing}
Usability testing of the program took place between the 5th of October and 9th of October 2015. Users were asked to perform multiple tasks with in the system testing the full functionality. 

\section{The participants}
The participants used in the testing were largely students from different departments within the university. Each participant was asked to use the program to perform various types of recordings on all the types of models available. The feedback from each participant included their opinion on the following: accuracy of recordings, guided nature of the process, layout of the forms, preview window, size and style of font, size and style of buttons, loading times and calibration.

\section{The feedback}
The feedback taken from the users were gathered and then analysed. The results of the usability tests are listed below. The original system consisted of small text and blue colour back grounds.

\subsubsection{Calibration}
The calibration was considered easy to perform and the window that displays the position of the eyes was helpful in positioning the user for optimal use of the program. No updates were made to this form.

\subsubsection{Accuracy of recordings}
The heat-maps and gaze-plots have a high enough accuracy that most participants agreed with the results. There were a couple of inconsistencies between the gaze-plot and the heat-maps.

\subsubsection{Layout of the forms}
The forms were thought to be very simple and clean. Initially (with the first group of participants) the layouts were considered very cluttered. The text was hardly readable and the blue back colour was not to every ones liking. We updated this by making form text bigger and more readable, as well as allowing users to change themes for the application. The base theme was also updated to a lighter more appealing style.

\subsubsection{Guided nature of the process}
This helped most of those who were lost at first although advanced users felt a bit held back in terms of being able to do as they wish when they wanted.

\subsubsection{Preview window}
This window was useful to the user as it showed a preview of their model and they could then decide to continue the recording or choose a new model.

\subsubsection{Size and style of font}
The text was too small and hard to read at times, this made it hard to use as people couldn't see the buttons labels. This was improved upon but changing the font and size.

\subsubsection{Size and style of buttons}
The buttons were not easily to distinguish from the form as the colours where the same colour. This made it hard to see and click the buttons.

\subsubsection{Loading times}
Most of the participants agreed that the loading times were too long with any video processing and the 3D rendering.

\subsubsection{The changes made}
Changes were made to the interface to make it more usable. The colour of the form was updated to a lighter theme to help with this. This makes the text easier to read and the font has also changed and made bold to make it stand out more. The buttons have borders around them to make them easily distinguishable from the background and this helps you know where the buttons are.
