The program has a main procedure which outlines the entire process involved in using the program an this procedure can be subdivided into many sub-procedures
\subsection{Main Procedure:Eye tracking}
This will start from the moment the program is initiated.The first subsection is the selection of the method to start the entire process.This sub-procedures is called project state.The following sub-procedures is the setting up of the the recording project.This sub-procedures is called recording setup.The final sub-procedures called eye tracking recording is where the recording can be performed and then can be navigated to the setup sub-procedures to start the process from that point.
\subsection{Sub Procedure:Project Start}
This procedure is comprised of an option to start a new recording session from the beginning or choosing to continue from a previous session.Both these options will navigate to the next sub-procedures.The difference between the options is that creating a new session will create files and continuing a session will read from previously created files. 
\subsection{Sub Procedure:Recording Setup}
The recording setup procedure allows the user to select the type of recording they wish to make and then also to choose a name for the recording.This sub procedure is the start point for the loop of activity when a recording has been completed.
\subsection{Sub Procedure:Eye Tracking Recording}
This procedure starts with the calibration of the camera to the user.Then the user will start the recording and then data will then be analysed.After the recording has taken place the user can then choose to create the heatmaps,create statistical analysis or to then start a new recording.When starting a new recording then the recording setup procedure will be executed until the user exits the recording. 