\subsection{Scope and Purpose}
The purpose of this eye-tracking software is to provide the user with a eye-tracking software which will allow for the tracking of the eye on 2D and 3D models and videos. The system is centred around making eye-tracking possible for all media types.

The Geo-Informatics Department of the University of Pretoria wanted us to develop this software to help enable further research into how people view geological locations and what stands out to them when they view it. That is why we had decided to give the option of three different types of eye-tracking; namely video, two dimensional and three dimensional. This software will thus aid them in their research and help provide the data that they need to preform greater research in this field.

\subsection{Process Overview }
The Eye-tribe as a system comes with the infra-red camera which is used to track the eyes as well as a SDK that will allow you to develop programs to be used with the camera. There is also access to a few sample programs on the Eye-Tribe website. These can be downloaded and then they can be used with the camera to see all the functionality their devices provide.

These items can be found when purchasing The Eye-tribe camera. They also detail the full capabilities and features of the camera on there website. We have taken these elements and expended them and integrated them with aspects such as 3D modelling and enabled eye-tracking on this and other media formats.

There are a few core process that have been implemented and that will allow for more effective eye-tracking across mediums. The process are as follows:
\begin{itemize}
\item Eye tracking on 2D models, 3D models and videos.
\item Creation of heat maps from eye-tracking data.
\item Saving heat map on specific media.
\item Creating Statistics on information gathered.
\end{itemize}

The eye-tracking on the various models uses the camera to track the eyes of the user and then saves the data into a file so that it can be processed at any later stage.

The data that is saved is then used to create heat-maps. The heat-maps come in two forms: as an overlay and as an overlay applied over the image. The overlay shows the points of the heat-map on a blank background. This can then be over laid onto the desired media and shows where on the user has viewed the model.

The recorded data is also used to create a statistical report of the test. This report includes meta-data about the media recorded and then actual data from the stored points. The data on the recorded points includes data such as time, points gathered and a map that shows which sections where most looked at. The report is created in PDF format and although it does not have the most conclusive data we hope to improve it in the future to over a very in depth analysis and aid researchers in making conclusions and decisions.

The following process listed are the core of this program as they carry out the basic and most important functionality of this software.
