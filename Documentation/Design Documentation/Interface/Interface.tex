The interface design for this project is meant to be kept as simple as possible so that users could easily familiarize themselves with the program.The easy design will make the program more friendly and welcoming to the user.

\subsection{Evolution of the design design}
The previous design for the interface was a very primitive windows design.It was a basic and boring design.The forms would constantly open and close to new sections and was very user-friendly.The colours of the form were dull and lacking and appealing factor.The placement of elements were also not ideal for efficiency and clarity when using the program.\\

The new design improves greatly with a simple look interface and easy flowing from one form to the next.The placements of elements also uncluttered the interface to make it easier for the users to use the system.The colours have changed and made much better looking.
\subsection{Flow of design}
An important factor in the design of the interface design is the flow of the interfaces.The flow of interfaces can make a program easy to navigate or harder.We have tried to make it as easy as possible by flowing one form straight into the other in a logical order.The user will also be able to go back to previous forms.
\subsection{Form element placement}
The placements of all the elements are carefully placed so that the user is able to easily see how to do what actions.This is vital as this makes it easier to perform tasks and understand how to do these tasks.The elements are centred to the form so that it would be the first thing the user will see.The recording forms then have the buttons listed vertically on the right and a open space to the left that will be used to display the model that recording is happening on.

\subsection{Interaction with functionality}
The interfaces serve as a front end to the user and then functionality is added to the various elements of the form on the back end.This is how the Interfaces interact with the functionality.