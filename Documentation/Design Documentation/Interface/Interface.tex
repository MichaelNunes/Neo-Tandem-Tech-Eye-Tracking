The interface design for this project is meant to be kept as simple as possible so that users could easily familiarize themselves with the program. The design should minimize any mistakes the user could make while using the system. We have taken multiple user aspects into account to try and ensure this such as limiting what actions can be made at different times, ensuring no important steps are missed along the way.

\subsection{Evolution of the design design}
The previous design for the interface was a very primitive windows design .It was a basic and boring design which was only used as a proof of concept. The forms would constantly open and close to new sections and was not very user-friendly. The colours of the form were bland and lacking any appealing factors. The placement of elements were also not ideal for efficiency and clarity when using the program.\\
\\
The new design improves greatly with a simple interface design that minimizes user error and allows for an easy flow from one task to the next to the next. The placements of elements also uncluttered the interface to make it easier for the users to use the system. The application styles such as colours and texts have also been improved increase appeal.

\subsection{Flow of design}
An important factor in the design of the interface design is the flow of the interfaces. The flow of interfaces can make a program easy to navigate or harder. We have tried to make it as easy as possible by flowing one form straight into the other in a logical order. The user will also be able to go back to previous forms without any detrimental effects. The main goal behind the new designs was to create a work space for the user that would never be cluttered with options or additional forms. 

\subsection{Form element placement}
The placements of all the elements are carefully placed so that the user would be able to view what options are available. This is vital as this makes it easier to perform tasks and understand what tasks must be preformed before another can take place. when opening or creating a project elements are centred to the form so that it would be the main focus. The recording forms have the buttons listed vertically on the right and a open space to the left that will be used to display a preview of the model that is chosen by the user.

\subsection{Interaction with functionality}
The interfaces serve as a front end to the user and functionality is added to the various form elements which allow relevant application functionality to be called. 