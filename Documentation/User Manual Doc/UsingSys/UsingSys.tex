The following section will go in to in depth detail about the system and how to use it. This section of the user manual will show you how to perform actions throughout the application. There will also be sections on what to expect from each action. Certain troubleshooting sections will be referenced in this text and if you encounter any of these errors please refer to the troubleshooting section to try and resolve the problem.

\subsection{Starting the program}
The program will allow you to perform eye-tracking on different kinds of media. Different media can be recorded and so the program will allow for multiple recordings' of the same or different media. The program allows you to create what is known as a eye-tracking recording project. This project is used to keep all the information of your recordings' for a specific project. You can always return to the session and continue recording in the desired project if required. Switching of projects is possible and thus allows you to easily switch between recording projects or create new ones.\newline

\includegraphics[width=400px,height=200px]{./Images/StartupPage.PNG}

When the user runs the Eye tracking software a selection form will appear. The form contains two options, namely start a new recording or to open a existing project. Below are the execution of both methods.

\subsubsection{Start a new project} 
To start a new project you will need to specify the name for the project and location to store all the files. A default name will be given to the project called "project" in the event that no name is created. This will create three special files and a series of folders in the selected location. The files created are the main project file, containing the ".eye", which contains all the information about the recording project. This information is all that is needed to continue recordings'. The two other files that are created are the settings files used to record general user settings and settings for three dimensional recordings. When the project is created a new directory will be created named "Recordings". With in this directory multiple sub-directories will be created detailing each model types name and where all their data will be saved. The user will now be prompted to create a new recording detailed further below.

		\includegraphics[scale=0.2, width=10cm, keepaspectratio]{./Images/NewProject.PNG}

\subsubsection{Open existing project} 
This will allow you to select an existing project and continue recording. Press the open button to select a file. The only file that will be accepted are ".eye" files. Once you have selected the file the settings are loaded the program moves to the work space to continue the recording process.\\
		\includegraphics[scale=0.2, width=10cm, keepaspectratio]{./Images/OpenProject.PNG}\\
		
\iffalse
\subsection{Menu Screen}
The menu screen is where the program will allow the user to navigate through the entire application and perform all the tasks. The menu is divided into two sections. The calibration and the recording sections. The calibration button will navigate the user to the configurations page and will allow the user to calibrate The Eye Tribe camera. This will need to be done to ensure that the camera is recording the data accurately. There is more detail on the calibration below.

\includegraphics[width=400px,height=200px]{./Images/Mainpage.JPG}

The recording option will navigate to the recording set-up form which will be used to set up a recording. This will be discussed later on in the manual. While the menu screen offers navigation for the application it also provides a exit point for the application.This is where the the program is exited. This is achieved by pressing the red X in the top right corner of the form. This will safely end the program and ensure that nothing harms the computer.
\fi

\subsection{Recording Set-up Screen}
When you are done opening an existing project or starting a new one you will be navigated to the screen shown below. This screen is used to set-up the recordings'. This screen will allow you to do multiple recordings' of different types. These recordings' can not be performed at the same time and thus only one type can be chosen at a time. Models can be chosen by navigating through the model types using the right and left arrows. You will also be allowed to name the recordings, this will create new folder in the recordings directory. The folder will be placed in the appropriate recording type sub-folder of the recordings folder. The folder will house all the information about the recordings such as eye-tracking data and also the resulting heat maps. 

\includegraphics[width=400px,height=200px]{./Images/ModelSelect.PNG}

The screen has a text box for the user to enter the name of the recording. The name should contain only alphanumeric characters so that when the folder is created it will not fail. If it does not follow these rules then it will not be allowed to continue to the next section of the application. Once ready then press the "Create Recording" button to proceed to the recording preview page. You will be greeted to either of three screens: 2D recording screen, 3D recording screen or the video recording screen with a host of option on the right side of the form.


\subsection{Calibration Screen}
When navigating to a recording screen there will be a button to navigate to the calibration screen. The calibration of The Eye Tribe camera is vital as this will ensure that the program will accurately create results for the media that the recording is performed on. This will also allow the user to configure the settings of the recording to their needs. The pressing of the calibration button will open the "EyeTribeUIWin" program which initiates the server and opens up the calibration. Use the calibration to calibrate the camera. Leave the server window open as this is what will transfer the data from the camera to the program. Once you have calibrated you may then proceed back to the menu screen and then start recording.
\\ \\ 
The following screen should appear if camera is connected correctly.\\
\includegraphics[width=400px,height=200px]{./Images/EyeCalibration.PNG}\\

The calibration will show the following if there is no camera detected.\\
\includegraphics[width=400px,height=200px]{./Images/NoCameraCalibration.PNG}\\

The calibration will show the following if there is no eyes detected.\\
\includegraphics[width=400px,height=200px]{./Images/NoEyeCalibration.PNG}\\

\subsection{Choosing model}
The user must select the model they wish to do recording on by pressing the choose model option. This will open up a dialogue for the user to select their model. Note that if a 2D recording is chosen then it will not be possible to choose a model of a different type such as video and vice-versa for all model types that are catered for.

\includegraphics[width=400px,height=200px]{./Images/ChooseModel.PNG}

\subsection{Recording 2D model}
The 2D model recording form is navigated after the 2D model is selected in the recording set up. The page has a list of buttons that will perform all the tasks that need to be performed. The open model button when pressed, will allow the selection of a 2D model to be chosen. The 2D models are often just images and any image is allowed to be chosen. The quality of the image will improve the results from the eye tracking recording. When an image is selected then the record button can be pressed. When the record button is pressed, the screen is filled with the image. The image is expanded to fit on the entire width and height of the screen and thus an image of higher quality will provide a crisper and more clear image. The recording also starts once the image is made full screen. During this process it is recommended that the subject moves as little as possible as the data that is recorded could be tampered with or come out incorrectly. Once you are done with the recording press the escape (Esc) key. You are then able to find the results of the recording in the project location in the relevant recordings folder, and finding the recording name as a folder. When you are done recording the data you can then exit the application or restart the recording. Restarting the recording will continue the recording for that model. The exiting of the application is done through the pressing the red X in the top right corner of the form.\\

Model displayed in the window and recording is ready.\\
\includegraphics[width=400px,height=200px]{./Images/ModelPreview.PNG}\\

The recording occurring on a model, 2D in this case.\\
\includegraphics[width=400px,height=200px]{./Images/ModelDisplay.PNG}\\

\subsection{Recording 3D model}
The 3D model recording form is navigated after the 3D model is selected in the recording set-up. The page has a list of buttons that will perform all the tasks that need to be performed. The open model button when pressed,will allow the selection of a 3D model to be chosen. The 3D model will need to be in the format of a object file. These files have the extension ".obj". When an 3D model is selected then the record button can be pressed. When the record button is pressed. The 3D model then has snapshots taken and a slide-show is created of the model that can then be shown. The slide-show is expanded to fit on the entire width and height of the screen. The recording also starts once the slide-show the images are made full-screen. During this process it is recommended that the subject moves as little as possible as the data that is recorded could be tampered with or come out incorrectly. The recording process will end when all the slide-show images have been shown.You are then able to find the results of the recording in the project location in the relevant recordings folder, and finding the recording name as a folder. When you are done recording the data you can then exit the application or restart the recording. Restarting the recording will continue the recording for that model. The exiting of the application is done through the pressing the red X in the top right corner of the form.

\subsection{Recording 3D model flythrough}
The 3D model fly-through recording form is navigated after the 3D model is selected in the recording set-up. The page has a list of buttons that will perform all the tasks that need to be performed. The open model button when pressed, will allow the selection of a 3D model to be chosen. The 3D model will need to be in the format of a object file. These files have the extension ".obj". When an 3D model is selected then the record button can be pressed. When the record button is pressed. The 3D model then is rendered and then the user is placed inside of the model and is then free to roam the model using either the keyboard or a game-pad. The 3D render is expanded to fit on the entire width and height of the screen. The recording also starts once the 3D render starts the is made full-screen. During this process it is recommended that you move as little as possible as the data that is recorded could be tampered with. The recording process will end when all the slide-show images have been shown.You are then able to find the results of the recording in the project location in the relevant recordings folder, and finding the recording name as a folder. When you are done recording the data you can then exit the application or restart the recording. Restarting the recording will continue the recording for that model. The exiting of the application is done through the pressing the red X in the top right corner of the form.

\subsection{Recording Video}
The video recording form is navigated after the video is selected in the recording set-up. The page has a list of buttons that will perform all the tasks that need to be performed. The open video button when pressed, will allow the selection of a video to be chosen. The video that is chosen needs to be a ".wmv" file as it uses and extension of the Windows Media player. The length of the length of the video can be any length but the longer the video is the longer it will take to be processed. When an video is selected then the record button can be pressed. When the record button is pressed, the screen is filled with the video and the video will then start playing. The video is expanded to fit on the entire width and height of the screen. The video is recommended to be a medium quality one as the video might show signs of pixelation if it is any lower. The recording also starts once the video is made full screen and begins to play. During this process it is recommended that you move as little as possible as the data that is recorded could be tampered with. The recording will end when the video has stopped playing. Once you are done with the recording process you are able to find the results of the recording in the project location in the relevant recordings folder, and finding the recording name as a folder. When you are done recording the data you can then exit the application or restart the recording. Restarting the recording will continue the recording for that model. The exiting of the application is done through the pressing the red X in the top right corner of the form.

\subsection{Report summary creation}
Once a recording has been completed on a model a statistical report is then generated with various details on the recording. The report is in a PDF format so a PDF reader such as Adobe reader is required to be installed on the computer to view the report. The report is located in the folder specified for the model. The statistical report contains three types pages. The first page type is a cover page with a generic heading, the second page type will contain data of the selected model. This information is basic meta-data such as the models name, the models location and various other details about the model recorded. The third and final page type contains all the data for the recording that is then summarised. This data includes the number of points used and how many points where know as "lost points". Lost points are points that have been discarded as they are not in the bounds of the screen. These points have no impact on the creation of the heat-maps. The data also tracks the time of the recording based on the amount of points that are recorded.\\
The pdf also contains a grid of where the points are mapped on the grid. The grid is a three by three(3X3) grid and each sections represent a part of a screen. The grid is also used to determine a point of interest which is a section that has been looked at the most. All this information is written to the PDF and then saved for later reading and to easily and quickly view what happened in the recording. An example of how a report may look is located below.

\includegraphics[width=400px,height=200px]{./Images/reporting.PNG}

\subsection{Heat Maps}
Heat maps show how the eye is moving around the recorded media by painting different colours at points that are being looked at. These points change colour the more often that they are viewed with red points being the most viewed. These create results that can be easily compared to others of the kind to see where users have looked most and how attractive the point was compared to any others. 

\includegraphics[width=400px,height=200px]{./Images/hm.JPG}
\subsection{Gaze Plot Maps}
Gaze plot maps show how the eye is moving around the recorded media. This records the fixations in the eyes as the move from one location to another. This differs from the heat-map in that is shows where the eyes fixate in order and for how long the subject fixates on a specific point by making use of circles are placed on the eyes location and will grow longer the eye is near to the same location. When the eye moves far enough away from the point it then stops growing and moves to the next point. This gaze plot can be created by pressing the create gaze plot button which then creates a video which can then be viewed.

\includegraphics[width=400px,height=200px]{./Images/ET.JPG}
 
\iffalse
\subsection{Settings}
There are many settings that the user is able to change for the program.These settings can be for aesthetics of the program itself and also for the actual recording process.These settings are saved and can be changed easily.Below are the settings and what each of them are for:
\subsubsection{Themes}
Themes will allow the users of the program to change the look and feel of the program.The user can choose a set of themes that will change how the form looks.At the moment there are only 3 available themes.They are: light which is white based background with black  text,Dark which offers a contrast with a dark background with white tex and finally there is NNT which incorporates the development teams colours which is a orange background with dark text. 
\subsubsection{3D recording settings}
There are settings for 3D model recording that allow the users to modify what is displayed.The textures setting allows the user to choose the if textures should be applied to the model.This can be checked for yes and be unchecked for no.The directional lighting setting allows the user to show directed light on the model.This setting should normally be on as it would make the model easier to see in most cases.This can be checked for yes and be unchecked for no
\subsubsection{Gaze Plot settings}
Gaze plot has two settings to change how these will look.The first is the toggling of showing numbers.Numbers will be shown next to points.These can be switched on by checking the box or off by deselecting the box.The second setting is number of points.This indicates how many points appear before the least recent ones start to fade.This number must be between 10 and 30.
\subsubsection{Recording settings}
Here there is one setting which is record time setting.This setting sets the length that an image will be shown for recording.The time is chosen in milliseconds and can be scaled up to 5 minutes.
\fi