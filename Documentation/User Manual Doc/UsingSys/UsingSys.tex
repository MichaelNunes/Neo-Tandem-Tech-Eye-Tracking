The following section will go in to in depth detail about the system and how to use it. This section of the user manual will show you how to perform actions throughout the application.There will also be sections on what to expect from each action.Certain troubleshooting sections will be referenced in this text and if you encounter any of these errors please refere to the troubleshooting section to try and solve the error.

\subsection{Starting the program}
The program will allow you to perform eye-tracking on different kinds of media.Different media can be recorded and so the program will allow for multiple recordings of the same or different media.The program allows you to create what is known as a eye-tracking recording project.This project is used to keep all the information of your recordings for a specific session.You can always return to the session and continue recording in that project which allows you to move between recording projects.\newline

 When the user runs the Eye tracking software a form will appear.The form is split into two sections, start a new recording or to open a existing project.Below are the execution pf both methods

\subsubsection{Start a new project} 
To start a new project you will need to specify a name for the project and location to store all the files.A default name will be given to the project called "project".This will create 3 files and a series of folders in the location selected.The 3 files created are the .eye file which contains all the information about the recording project.This information is all that is needed to continue recordings.The two other files that are created are the settings files for general user settings and then settings for 3D recording.The folders that are created are just the folders for the recording data and the results.Once this all the folders are created then the program will proceed to the main menu.
\subsubsection{Open existing project} 
This will allow you to select an existing project and continue recording.Press the open button to select a file.The only file that will be accepted are .eye files.Once you select the file the settings are loaded up from the files and then the program moves to the main menu for you to continue recording.
