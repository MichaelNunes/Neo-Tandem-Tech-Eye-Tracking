\subsection{Render 3D scene}
    \textbf{Important}\newline
    Given a 3D scene, in a format such as a blend file or an obj file, the scene must be able to be rendered on a display. The rendered scene will be placed in a window for viewing so that the eye-tracker can "see" the scene.
\subsection{Eye-track on a 3D scene}
    \textbf{Important}\newline
    The eye-tracking software must be used or augmented in such a way that specific areas in the 3D scene can capture viewing information. This is will be done given information from the eye-tracker itself.
\subsection{Create heat-mapped texture}
    \textbf{Important}\newline
    From the eye-tracking information a heat map in the form of a texture map must be produced.
\subsection{Map heat-mapped texture back onto scene}
    \textbf{Important}\newline
    The heat-mapped texture of the object must be able to be mapped correctly back onto the scene, that is the placement must be as was the original texture.
\subsection{Dual texture mapping}
    \textbf{Nice-to-have}\newline
    Both the original and the heat-mapped texture are visible in the scene at the same time.
\subsection{Interactive Maps}
Description: This function involves overlaying a map with data captured by the Eye tribe eye tracking technology and stored using the OGAMA freeware. The data will be retrieved from a database which has already been integrated into the OGAMA freeware that will be used for this project. Through the use of the analytical features provided by OGAMA, its functionality will be extended into the visualization of the information it has stored. This includes the attention maps, scan paths and replay functions.
Priority: Important
Preconditions: Have the Eye tribe eye tracking technology running during the activity.     
Post Conditions: Output a 2D representation on the interactive map with the necessary data overlaying it. 
