This section of the user manual is dedicated to showing the errors that can occur on the system.This will also provide possible solutions to problems that you may encounter. Please note that version 1.0, our first stable release, should have a concise guide of errors you may encounter during using the application. If you find any errors, that are not mentioned in this documentation then please do not hesitate to contact Neo Tandem Technologies and we will do our best to try and solve the issue.

The system is split into many subsections thus the section below has also been broken up to try ease the process of solving problems and queries.

\subsection{Starting the program}
\subsubsection{String is not a path}
This error occurs when starting a new recording project and the path in the path text-box does not follow a natural path structure. The program will not continue if this error occurs and is not fixed.
This can be fixed by ensuring that the text inside is a actual path such as: C:/User/[MyPath].

\subsubsection{The path does not exist}
The following error occurs when the path that is selected no longer exists on the computer. This can be resolved by doing one of two things:
\begin{enumerate}
\item Ensuring that the path specified is correct.
\item Creating the file path to match the selection. 
\end{enumerate}
Any one of these solutions can help fix the problem and will allow the user to continue normally.

\subsection{Calibration}
\subsubsection{Could not find EyetribeWinUI.exe}
The program requires that The Eye-Tribe calibration software be installed on the users system and thus occurs when the software can not be found on the system. The solution is simply to install the software which can be found on The Eye-Tribe website after you have purchased the camera.

\subsection{Naming}
\subsubsection{Name is to long or contains an invalid character}
This error can occur when the name for the recording contains invalid characters. All naming is required to only contain alphanumeric and thus can not contain any symbols. If this requirement has been met then please ensure that all names do not have any more than 40 alphanumeric characters.

\subsection{General recording errors}
\subsubsection{Could not locate file}
When a recording has been completed and this error occurs it means that the file that stores the eye tracking data may never have been created, thus it can not create heat maps for the desired model. This error can occur due to multiple factors such as:
\begin{enumerate}
\item The eye-tracking camera has not been connected or calibrated correctly.
\item The port used to connect the camera is faulty or does not comply with the specifications.
\end{enumerate}
The solution would simply be to check that The Eye-tribe is connected to the computer properly and also that it is connected to a USB 3.0 port.

\subsection{Heat-map}
\subsubsection{Heat-map could not be created}
The heat map is created using a text file created when a recording has been done. This text file is then used in the heat-map production process. This error occurs when the text file is no longer available. The solution for this is to restart the recording process and ensure that the text file appears in the recording directory. If the file does not appear in the directory then it is recommended that you restart the application, if this fails then please restart the computer.

\subsubsection{Video could not be created}
When the video is created an error can occur due to the following factors:
\begin{enumerate}
\item The required image is in use.
\item The required image has been deleted.
\end{enumerate}
The solution is to re-run the creation of the video and avoid doing the above so that the video can be processed and created properly. Note viewing created images or deleting any created files can potentially cause this error to appear as thus it is recommended to only view these files once completed.

\subsection{Statistics}
\subsubsection{Could not create statistical report} 
The statistical report is created using the recorded eye tracking data file created by the application after recording has been completed. This error occurs when the text file is no longer available when trying to create the report. The solution for this is to restart the recording process and ensure that the text file appears in the recording directory. If the file does not appear in the directory then it is recommended that you restart the application, if this fails then please restart the computer.

\subsubsection{Report could not be saved}
The report will be saved to the directory of the relevant model type. If you recreate the report an error can arise as it needs to save the report. The reason for this error is that the report is either open already or the directory has been deleted. This can be resolved by closing the report and then creating the report once more and ensuring that the directory still exists.