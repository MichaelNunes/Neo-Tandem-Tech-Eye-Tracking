This section of the user manual is dedicated to showing the errors that can occur on the system.This will also provide possible solutions to problems that you may encounter. Please note that currently the program is not finished and the accompanying user manual is also not yet complete. This will be updated as the system is developed.\\

The system is split into many subsections. This section will also be split to ensure you are able to file what you need.

\subsection{Starting the program}
\subsubsection{String is not a path}
This error occurs when starting a new recording project and the path in the path text-box does not follow path structure. The program will not continue if this error occurs.
This can be fixed by ensuring that the text inside is a actual path such as: C:/User/MyPath.
\subsubsection{The path does not exist}
The following error occurs when the path that is selected no longer exists on the computer. This can be fixed by doing on of two things:\\
\begin{enumerate}
\item Ensuring that the path specified is correct.
\item Creating the file path to match the selection. 
\end{enumerate}
Any one of these solutions can help fix the problem and allow the program to continue.
\subsection{Calibration}
\subsubsection{Could not find EyetribeWinUI.exe}
The program requires that the Eye-tribe software be installed on the system.This error occurs when the software is not found on the system.\\
The solution is simply to install the software which can be found on The Eye-Tribe website when you have purchased the camera.
\subsection{Naming}
\subsubsection{Name has invalid characters or too many}
This error occurs when the name for the recording contains invalid characters.These characters need to be alphanumeric and can not contain any symbols.\\
The solution is to simply use only alphanumeric numbers and not symbols.Use names with 40 characters.
\subsection{General recording errors}
\subsubsection{Could not find [insert path and name].txt}
When a recording has been completed and this error is shown it means that the file that stores the data was never created so it can not create heat maps for the specific media type.This error can occur due to multiple factors.\\
\begin{itemize}
\item The eye tracking camera was not connected or was not calibrated correctly.
\item The eye tracking camera was connected to a USB 2 port.
\end{itemize}
The solution would simply be to check that The Eye-tribe is connected to the computer properly and also that it is connected to a USB 3 port.
\subsection{Heatmap}
\subsubsection{Heatmap could not be created}
The heat map is created using a text file created by the program.This text file is then used in the heatmap making process.This error occurs when the text file is no longer available.\\

The solution for this is to restart the recording process and ensure that the text file appears in the directory.If the file does not appear in the directory then it is recommended that you restart the computer.
\subsubsection{Video could not be created}
When the video is created an error can occur due to the following factors:
\begin{itemize}
\item A image for recording is in use.
\item A image for recording has been deleted.
\end{itemize}

The solution is to re run the creation of the video and avoid doing the above so thatthe video can be processed and created properly.
\subsection{Statistics}
\subsubsection{Could not create statistical report} 
The statistical report is created using a text file created by the program.This text file is then used in the statistics making process.This error occurs when the text file is no longer available.\\

The solution for this is to restart the recording process and ensure that the text file appears in the directory.If the file does not appear in the directory then it is recommended that you restart the computer.

\subsubsection{Report could not be saved}
The report will be saved to the directory of the video model.If you recreate the report an error can arise as it needs to save the report.The reason for this error is that the report is either opened already or the directory was deleted.\\

This can be resolved by closing the report and then creating the report once more and ensuring that the directory still exists.